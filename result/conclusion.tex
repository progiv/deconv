\conclusion

В данной работе были рассмотрены методы деконволюции для восстановления изображений подвергнутых линейному или криволинейному смазу, способы определения параметров искажения. По результатам проведённых исследований для был выбран модифицированный метод Люси-Ричардсона, позволяющий получить высокое качество восстановления.

В работе был разработан метод оценки параметров криволинейного смаза, для применения его в задаче восстановления изображения. Результаты его работы представлены в главе 3.

Для проведения исследований в рамках работы были реализованы на языке python:
\begin{itemize}
	\item модифицированный алгоритм Люси-Ричардсона~\cite{richardsonLucyModifiedBiggs};
	\item алгоритм градиентного спуска;
	\item разработанные алгоритмы оценки линейного и криволинейного искажающего оператора.
\end{itemize}

В дальнейшем предлагается исследовать модификацию предложенного алгоритма с нормализацией целевой функции, применение других методов минимизации, устойчивых к наличию локальных минимумов.