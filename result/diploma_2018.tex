\documentclass[%
12pt,
bachelor,    % тип документа
%natbib,      % использовать пакет natbib для "сжатия" цитирований
subf,        % использовать пакет subcaption для вложенной нумерации рисунков
href,        % использовать пакет hyperref для создания гиперссылок
%colorlinks,  % цветные гиперссылки
%fixint,     % включить прямые знаки интегралов
%substylefile = bachelor.rtx
]{disser}%twoside, 

\usepackage[a4paper, includefoot, left=3cm, right=1cm, top=2cm, bottom=2cm, headsep=1cm, footskip=1cm]{geometry}

\usepackage[intlimits]{amsmath}
\usepackage{amssymb, amsfonts}

\usepackage[T2A]{fontenc}
\usepackage[utf8]{inputenc}
\usepackage[english,russian]{babel}
\ifpdf\usepackage{epstopdf}\fi
\usepackage[autostyle]{csquotes}
%\usepackage[toc]{appendix}

% Шрифт Times в тексте как основной
\usepackage{tempora}
% альтернативный пакет из дистрибутива TeX Live
%\usepackage{cyrtimes}
%\usepackage{pscyr}

% Шрифт Times в формулах как основной
%\usepackage[varg,cmbraces,cmintegrals]{newtxmath}
% альтернативный пакет
%\usepackage[subscriptcorrection,nofontinfo]{mtpro2}

\usepackage[%
style=gost-numeric,
backend=biber,
language=auto,
hyperref=auto,
autolang=other,
sorting=none
]{biblatex}
\addbibresource{biblio.bib}

% Плавающие рисунки "в оборку".
\usepackage{wrapfig}

% Номера страниц снизу и по центру
%\pagestyle{footcenter}
\chapterpagestyle{footcenter}

% Точка с запятой в качестве разделителя между номерами цитирований
%\setcitestyle{semicolon}

% Использовать полужирное начертание для векторов
\let\vec=\mathbf

% Включать подсекции в оглавление
\setcounter{tocdepth}{2}

\graphicspath{{../pic/}}

\usepackage{import}
\usepackage{comment}
\usepackage{textcomp}
\usepackage{tikz}
\usetikzlibrary{shapes,arrows}

\usepackage{setspace}
\onehalfspacing

% вставлять используемые пакеты сюда
% hyperref должен быть самым последним, иначе будут  баги с \chapter \section и \subsection

%\usepackage[
%pdfstartview   = FitH,
%bookmarksopen  = true,bookmarksopenlevel=0,
%plainpages     = false,pdfpagelabels,
%pagebackref    = false,
%pdftoolbar     = false,
%unicode,
%pdftex,
%hidelinks
%]{hyperref}

% Русские больше и меньше либо равно
\renewcommand{\leq}{\leqslant}
\renewcommand{\geq}{\geqslant}
\DeclareMathOperator*{\argmax}{arg\,max}
\DeclareMathOperator*{\argmin}{arg\,min}
\DeclareMathOperator*{\conv}{*\!*}

\newtheorem{definition}{Определение}
\newtheorem{algorithm}{Алгоритм}

% Больше переносов, меньше overflow hbox
%\tolerance=1000
%\hyphenpenalty=1000

\begin{document}
    \institution{%
        МИНОБРНАУКИ РОССИИ \\
        Федеральное государственное автономное образовательное учреждение высшего образования\\
        <<Национальный исследовательский университет\\
        <<Московский институт электронной техники>>\\
        Факультет Микроприборов и технической кибернетики\\% (МПиТК)
        Кафедра высшей математики №1% (ВМ-1)
    }
\title{Бакалаврская работа}
\topic{Оценка линейного искажающего оператора в задаче \\ восстановления изображения}%\normalfont\scshape 

\coursenum{01.03.04}
\course{Прикладная математика}
%\masterprog{Математические методы и моделирование\\ \>\ \ в естественнонаучной и технической сферах}
\author{~Терентьев И.~В.}
\group{ВМ-40}
\authorFullName{Терентьев Иван Владимирович}
%\apname{Прокофьев А.А.}

%Научник
\sa {Умняшкин С.~В.}
\sastatus{профессор, д.ф.-м.н.}
\city{Москва}
\date{2018}

\maketitle

%{\fontsize{13pt}{17}}
\tableofcontents
% MAIN BODY
\subimport{.}{intro.tex}
\subimport{.}{chapter1_theory.tex}
\subimport{.}{chapter2_experiments.tex}
\subimport{.}{chapter3_results.tex}
\subimport{.}{conclusion.tex}

%\bibliographystyle{ugost2008}
%\bibliography{biblio}
% Список литературы
\printbibliography[heading=bibintoc]

% Приложения
\subimport{.}{attachments.tex}

\end{document}
