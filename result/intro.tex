\intro
В 2017 году у каждого второго в России есть смартфон\cite{mobileUsers}. 
Это значит, что у каждого второго россиянина есть маленькая камера, с немалыми вычислительными ресурсами. Из-за несовершенства фото-видео техники, а именно малой светосилы приходится увеличивать выдержку, в следствие чего при движении объекта или камеры на изображении возникает смаз. Это может стать причиной снижения информативности кадра. Кроме того широкое распространение получили экшн-камеры, которые специально созданы для работы в движении.

Устранять проблему можно аппаратно: увеличивая площадь матрицы и светосилу объектива или программно~--- применяя алгоритмы восстановления изображений. Первый подход не применим для мобильных устройств, так как увеличивает размеры, вес и стоимость аппарата. Второй применить с каждым годом становится проще, так как растут вычислительные мощности устройств. Поэтому задача компенсации смаза является актуальной.
Пользователи смартфонов редко используют штативы для съёмки, поэтому аппарат испытывает тряску, как следствие изображение будет смазано. При этом в общем случае движение нелинейно и может быть представлено кривой линией.

В данной работе будет использован итерационный метод Люси-Ричардсона для восстановления изображений и метод оценки линейного искажающего оператора на основе кепстра изображения. Будут предложены дополнения и модификации алгоритма для более точной реконструкции искажённого кадра. Также рассмотрен алгоритм оценки криволинейного искажающего оператора с модификациями.

Целью данной работы является разработка метода компенсации криволинейного смаза изображения, основанного на методе Люси-Ричардсона. Для достижения этой
цели необходимо поставить и решить следующие задачи:

\begin{itemize}
\item разработать метод оценки параметров искажения;
\item определить критерий качества восстановленного изображения и искажающего оператора
\item реализовать алгоритм, который будет принимать на вход искажённое изображение и возвращать оценку неискажённого изображения.
\end{itemize}

В первой главе рассмотрены теория, описывающая модель искажения, методы оценки его параметров и его устранения.

Во второй главе описываются программные решения, с помощью которых происходит оценка качества работы алгоритмов, определение параметров смаза.

В третьей главе приводятся результаты экспериментов. Определяется модифицированный метод Люси-Ричардсона на основе проведённых экспериментов.
