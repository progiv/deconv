\documentclass[14pt, bachelor, substylefile = bachelor.rtx]{disser}%twoside, 

\usepackage[a4paper, includefoot, left=3cm, right=1cm, top=2cm, bottom=2cm, headsep=1cm, footskip=1cm]{geometry}

\pagestyle{footcenter}
\chapterpagestyle{footcenter}

\usepackage[T2A]{fontenc}
\usepackage{ucs} 
\usepackage[utf8x]{inputenc}
\usepackage[english, russian]{babel}
\ifpdf\usepackage{epstopdf}\fi
\setcounter{tocdepth}{2}

\usepackage{comment}
\usepackage{amssymb}
\usepackage{amsmath}
\usepackage{textcomp}
\usepackage{tikz}
\usetikzlibrary{shapes,arrows}

% вставлять используемые пакеты сюда
% hyperref должен быть самым последним, иначе будут  баги с \chapter \section и \subsection

\usepackage[
pdfstartview   = FitH,
bookmarksopen  = true,bookmarksopenlevel=0,
plainpages     = false,pdfpagelabels,
pagebackref    = true,
pdftoolbar     = false,
unicode,
pdftex,
hidelinks
]{hyperref}

\renewcommand{\leq}{\leqslant}


\begin{document}
    \institution{%
        МИНОБРНАУКИ РОССИИ \\
        Федеральное государственное автономное образовательное учреждение высшего образования\\
        <<Национальный исследовательский университет\\
        <<Московский институт электронной техники>>\\
        Факультет Микроприборов и технической кибернетики (МПиТК)\\
        Кафедра высшей математики №1 (ВМ-1)
    }
\title{Бакалаврская работа}
\topic{Оценка линейного искажающего оператора в задаче \\ восстановления изображения}%\normalfont\scshape 

\coursenum{01.03.04}
\course{Прикладная математика}
%\masterprog{Математические методы и моделирование\\ \>\ \ в естественнонаучной и технической сферах}
\author{Терентьев И.~В.}
\group{ВМ-40}
\authorFullName{Терентьев Иван Владимирович}
%\apname{Прокофьев А.А.}

\sa {Умняшкин С.~В.}l
\sastatus{профессор, д.ф.-м.н.}
\city{Москва}
\date{2018}

{\fontsize{11pt}{14pt}\maketitle}

{\fontsize{12pt}{14}\tableofcontents}

\intro

Льем воду.

Так как нужно 25-30+ страниц.

А это нелегко без нее.

\chapter{Обзор проблематики восстановления изображений}

\section{Модель искажения}
Изображение~--- это двумерная проекция трёхмерной сцены. С помощью записывающей системы(камеры) она проецируется на двумерную область~--- изображение. Под изображением понимаем двумерный дискретный сигнал $f(x,y)$, где $0\leq x \leq M$, $0\leq y\leq N$ ($M$, $N$~--- ширина и высота изображения соответственно). В данной работе рассматриваем только обработку полутоновых изображений со значениями яркости $0\leq f(x,y)\leq 1$.

\begin{figure}[h]
	\tikzset{%
		block/.style    = {draw, thick, rectangle, minimum height = 3em,
			minimum width = 3em, node distance = 3cm},
		sum/.style      = {draw, circle, node distance = 3cm}, % Adder
		input/.style    = {coordinate}, % Input
		output/.style   = {coordinate, node distance = 3cm} % Output
	}
    \centering
    \begin{tikzpicture}[auto, thick, node distance=2cm, >=triangle 45]
    \draw
        node [input, name=in] {} 
        node [block, right of=in] (hxy) {$h(x,y)$}
        node [sum, right of=hxy] (sum1) {\Large$+$}
        node [input, above of=sum1] (noize) {}
        node [output, right of=sum1] (out){$g(x,y)$};
    \draw[->](in)     -- node {$f(x,y)$}(hxy);
    \draw[->](hxy)    -- node {$s(x,y)$}(sum1);
    \draw[->](noize)  -- node {$\eta(x,y)$}(sum1);
    \draw[->](sum1)   -- node {$g(x,y)$}(out);
    \end{tikzpicture}
    \caption{Линейная система искажения изображения}
    \label{fig:distortionScheme}
\end{figure}

Пусть $f(x,y)$~--- неискажённое изображение; $g(x,y)$~--- изображение подвергнутое искажению; $h(x,y)$~--- импульсная характеристика (ИХ) оператора искажения; $\hat{f}(x,y)$~--- оценка неискажённого изображения $f(x,y)$; $\eta(x,y)$~--- некоррелированный гауссов шум. Обозначим $F(u,v)$, $G(u,v)$, $H(u,v)$, $N(u,v)$ Фурье-образы функций $f(x,y)$, $g(x,y)$, $h(x,y)$ и $\eta(x,y)$ соответственно, полученные дискретным преобразованием Фурье\cite[стр.~332]{gonsalesDigital2012}.
Допускаем, что процесс формирования искажённого изображения $g(x, y)$ линейный и его можно описать с помощью линейной дискретной системы \cite[стр.~403]{gonsalesDigital2012}
\begin{equation}\label{eq:distortion}
g(x, y) = h(x,y) *\!*f(x,y) + \eta(x,y),
\end{equation}

где $<*\!*>$~--- операция двумерной свёртки изображения $f(x,y)$ размером $M\times N$ с импульсной характеристикой фильтра $h(x,y)$ размером $m\times n$, которая описывается следующим выражением~\cite[стр.~298]{gonsalesDigital2012}:

\begin{equation}
h(x,y) *\!* f(x,y) = \sum_{s=-a}^{a}\sum_{t=-b}^{b}h(s,t)f(x-s, y-t),
\end{equation}
где $a=(m-1)/2$ и $b = (n-1)/2$.

\begin{definition}
Двумерная импульсная характеристика (или функция рассеяния точ-
ки, ФРТ) — это реакция двумерной дискретной системы на единичный импульс.
$$\delta(n,m) = 
	\begin{cases}
		1, n=m=0\\
		0, n\ne 0, m\ne 0
	\end{cases}$$.
\end{definition}

Таким образом, процесс искажения изображения можно описать как результат взаимодействия исходного изображения $f(x, y)$ с линейно-дискретной системой, изображённой на рисунке~\ref{fig:distortionScheme}. Сигнал $f(x, y)$ подвергается воздействию оператора смаза с импульсной характеристикой $h(x, y)$ , которая обычно не известна заранее. Из-за внешних факторов и несовершенства съёмки к искажению добавляется шум, который считается некоррелированным случайным процессом, также некоррелированным с изображением.
В частотной области процесс искажения (\ref{eq:distortion}) выглядит следующим образом:

\begin{equation}\label{eq:distortionFourier}
G(u,v) = H(u,v)\cdot F(u,v) + N(u,v),
\end{equation}

так как операция свёртки в пространственной области эквивалентна умножению в частотной области\cite[стр.~39]{basicsOfDigitalDataProcessing2016Umnyashkin}

Смаз на изображении всегда возникает при относительном движении камеры и объекта. Для определения импульсной характеристики оператора искажения, рассмотрим случай, когда камера перемещается с постоянной горизонтальной скоростью относительно сцены. В дискретном случае~\cite{iterableImageRestorationBiemonLangddeik}:
\begin{equation}\label{eq:horizontalBlurPsf}
	h(x,y) = 
		\begin{cases}
			\frac{1}{L+1}, & 0 \leq y \leq L, x=0\\
			0              & \text{в остальных случаях}
		\end{cases},
\end{equation}

где L~--- величина смаза, то есть количество точек(пикселей), на которое сместилось изображение в , соответствующих одной точке объекта. Выражение (\ref{eq:horizontalBlurPsf})~--- один из вариантов функции рассеяния точки.

Частотная характеристика данного оператора искажения определяется выражением~\cite{iterableImageRestorationBiemonLangddeik}:

\begin{equation}\label{eq:horizontalBlurIRFourier}
H(u,v) = \frac{1}{L+1}e^{-i\frac{L\pi}{N}m}\frac{\sin\frac{\pi(L+1)u}{N}}{\sin\frac{\pi u}{N}}
\end{equation}


\section{Методы восстановления изображения}

\subsection{Инверсная фильтрация}
\subsection{Фильтр Винера}
\subsection{Регуляризация по Тихонову}
\subsection{Метод Люси-Ричардсона}
\subsection{Слепая деконволюция}


\section{Постановка задачи оценки искажающего оператора}



\conclusion

Подводим заключение всей нашей эпопее по написанию ВКР.

Также в данном пункте можно вставить благодарности людям, помогавшим в написании работы: маме, дяде Феде с соседнего подъезда и коту.

\bibliographystyle{gost2008}
\bibliography{biblio}

\end{document}

%А в работе \cite{Allen} утверждалось ..., что является противоречием к работе \cite{ARIMA}

% пример цитирования
%В работе \cite{ARIMA} было сказано ...
