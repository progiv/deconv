\documentclass[14pt, bachelor]{disser}

\usepackage[a4paper, includefoot, left=3cm, right=1cm, top=2cm, bottom=2cm, headsep=1cm, footskip=1cm]{geometry}

\pagestyle{footcenter}
\chapterpagestyle{footcenter}

\usepackage[T2A]{fontenc}
\usepackage[utf8x]{inputenc}
\usepackage[english, russian]{babel}
\ifpdf\usepackage{epstopdf}\fi
\setcounter{tocdepth}{2}

% вставлять используемые пакеты сюда
% hyperref должен быть самым последним, иначе будут  баги с \chapter \section и \subsection

\usepackage[
pdfstartview   = FitH,
bookmarksopen  = true,bookmarksopenlevel=0,
plainpages     = false,pdfpagelabels,
pagebackref    = true,
pdftoolbar     = false,
unicode,
pdftex,
hidelinks
]{hyperref}




\begin{document}
    \institution{%
        Министерство образования и науки Российской Федерации \\
        Федеральное агентство по образованию \\
        Федеральное государственное автономное образовательное учреждение\\ высшего образования «Национальный исследовательский университет \\«Московский институт электронной техники»\\
        Факультет МПиТК\\
        Кафедра ВМ-1
    }
%\title{ВЫПУСКНАЯ КВАЛИФИКАЦИОННАЯ РАБОТА}
%\title{ДИССЕРТАЦИЯ\\[-14pt]на соискание ученой степени\\МАГИСТРА}
%\topic{\normalfont\scshape %
%    FIXME <<Название>>}
\topic{<<Оценка линейного искажающего оператора в задаче восстановления изображения>>}

\coursenum{01.04.04}
\course{Прикладная математика}
%\masterprog{Математические методы и моделирование\\ \>\ \ в естественнонаучной и технической сферах}
\author{Иван Терентьев}
\group{ВМ-40}

%\apname{Прокофьев А.А.}

\sa {Умняшкин С.В.}
\sastatus{Профессор, д.ф.-м.н.}
\city{Москва}
\date{2018}

\maketitle

\tableofcontents

\intro

Льем воду.

Так как нужно 60-80 страниц.

А это нелегко без нее.

\chapter{Название главы}

% пример цитирования
В работе \cite{ARIMA} было сказано ...

\section{Название секции}

А в работе \cite{Allen} утверждалось ..., что является противоречием к работе \cite{ARIMA}

\subsection{Название подсекции}

Длинный-длинный абзац, в котором рассказывается про отношение пишущего все это человека к существующему мироустройству, порядку вещей, главному вопросу жизни, вселенной и всего такого.




\conclusion

Подводим заключение всей нашей эпопее по написанию ВКР.

Также в данном пункте можно вставить благодарности людям, помогавшим в написании работы: маме, дяде Феде с соседнего подъезда и коту.

\bibliographystyle{gost2008}
\bibliography{biblio}

\end{document}
