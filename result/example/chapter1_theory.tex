\chapter{Теоретические сведения}

\section{Постановка задачи}

Изображение~--- это двумерная проекция трёхмерной сцены. В данной работе изображением называем двумерный дискретный сигнал $f(x,y)$, где $0\leq x \leq M$, $0\leq y\leq N$ ($M$, $N$~--- ширина и высота изображения соответственно). Рассматриваем только полутоновые изображения со значениями яркости $0\leq f(x,y)\leq 1.0$. Пусть $f(x,y)$~--- неискажённое изображение; $g(x,y)$~--- изображение подвергнутое искажению; $h(x,y)$~--- импульсная характеристика (ИХ) оператора искажения; $\hat{f}(x,y)$~--- оценка неискажённого изображения $f(x,y)$; $\eta(x,y)$~--- некореллированный гауссов шум.
Обозначим $F(u,v), G(u,v), H(u,v), N(u,v)$ Фурье-образы функций $f(x,y), g(x,y), h(x,y)$ и $\eta(x,y)$ соответственно, полученные дискретным преобразованием Фурье \cite[p.~284]{gonsalesDigital}

\begin{comment}
Изображение — это двумерная проекция трёхменой сцены, подлежащей съём-
ке. В данной работе под изображением мы понимаем двумерный дискретный сиг-
нал f (x, y) , где 0 ≤ x ≤ M , 0 ≤ y ≤ N ( M , N — ширина и высота
изображения соответственно). Рассматриваем только полутоновые изображения, где
0 ≤ f (x, y) ≤ 255 — яркость пикселя в точке с координатами (x, y) .
Пусть f (x, y) — неискажённое изображение; g(x, y) — искажённое изображе-
ние; h(x, y) — импульсная характеристика (ИХ) оператора искажения; f ^ (x, y) —
оценка изображения f (x, y) и η(x, y) — некоррелированный шум. Фурье-образы
функций f (x, y) , g(x, y) , h(x, y) , η(x, y) , полученных с помощью дискретного пре-
образования Фурье [1, c. 284], обозначим соответственно F (u, v) , G(u, v) , H(u, v)
и N (u, v) . Допускаем, что процесс формирования искажённого изображения g(x, y)
линейный и его можно описать с помощью линейной дискретной системы [1, c. 403]
(см. рис. 1.1):
\end{comment}


\subsection{Название подсекции}

Длинный-длинный абзац, в котором рассказывается про отношение пишущего все это человека к существующему мироустройству, порядку вещей, главному вопросу жизни, вселенной и всего такого.
