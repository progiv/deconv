\chapter{Обзор проблематики восстановления изображений}

\section{Модель искажения}

Изображение~--- это двумерная проекция трёхмерной сцены. С помощью записывающей системы(камеры) она проецируется на двумерную область~--- изображение. Под изображением понимаем двумерный дискретный сигнал $f(x,y)$, где $0\leq x \leq M$, $0\leq y\leq N$ ($M$, $N$~--- ширина и высота изображения соответственно). В данной работе рассматриваем только обработку полутоновых изображений со значениями яркости $0\leq f(x,y)\leq 1$.

\tikzset{%
  block/.style    = {draw, thick, rectangle, minimum height = 3em,
	minimum width = 3em, node distance = 3cm},
  sum/.style      = {draw, circle, node distance = 3cm}, % Adder
  input/.style    = {coordinate}, % Input
  output/.style   = {coordinate, node distance = 3cm} % Output
}
\begin{figure}
	\centering
	\begin{tikzpicture}[auto, thick, node distance=2cm, >=triangle 45]
	\draw
		node [input, name=in] {} 
		node [block, right of=in] (hxy) {$h(x,y)$}
		node [sum, right of=hxy] (sum1) {\Large$+$}
		node [input, above of=sum1] (noize) {}
		node [output, right of=sum1] (out){$g(x,y)$};
	\draw[->](in)     -- node {$f(x,y)$}(hxy);
	\draw[->](hxy)    -- node {$s(x,y)$}(sum1);
	\draw[->](noize)  -- node {$\eta(x,y)$}(sum1);
	\draw[->](sum1)   -- node {$g(x,y)$}(out);
	\end{tikzpicture}
	\caption{Линейная система искажения изображения}
	\label{distortionScheme}
\end{figure}

Пусть $f(x,y)$~--- неискажённое изображение; $g(x,y)$~--- изображение подвергнутое искажению; $h(x,y)$~--- импульсная характеристика (ИХ) оператора искажения; 
Тогда процесс искажения в пространственной области представим в виде
\begin{equation}\label{eq:distortion}
g(x, y) = h(x,y) *\!*f(x,y) + \eta(x,y)
\end{equation}
 $\hat{f}(x,y)$~--- оценка неискажённого изображения $f(x,y)$; $\eta(x,y)$~--- некореллированный гауссов шум.
Обозначим $F(u,v), G(u,v), H(u,v), N(u,v)$ Фурье-образы функций $f(x,y), g(x,y), h(x,y)$ и $\eta(x,y)$ соответственно, полученные дискретным преобразованием Фурье \cite[стр.~332]{gonsalesDigital}


\section{Методы восстановления изображения}

\subsection{Инверсная фильтрация}
\subsection{Фильтр Винера}
\subsection{Регуляризация по Тихонову}
\subsection{Метод Люси-Ричардсона}
\subsection{Слепая деконволюция}


\section{Постановка задачи оценки искажающего оператора}



\chapter{Метод оценки линейного искажающего оператора}

\section{Кепстральный метод оценки линейного оператора}
\subsection{Нахождение параметров искажения}
\subsection{Уточнение параметров искажения}

\section{Нахождение парметров криволинейного искажающего оператора}
\subsection{Пердставление криволинейного искажающего оператора}
\subsection{Первое приближение}
\subsection{Второе приближение}
\subsection{Уточнение параметров}
\section{Вывод}



\chapter{Результаты экспериментов}
\section{Оценка линейного искажающего оператора}
\section{Оценка криволинейного искажающего оператора}
\subsection{Искусственные искажения}
\subsection{Искажения <<от руки>>}
\section{Вывод}
