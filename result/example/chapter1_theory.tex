\chapter{Теоретические сведения}

\section{Постановка задачи}

Изображение~--- это двумерная проекция трёхмерной сцены. В данной работе изображением называем двумерный дискретный сигнал $f(x,y)$, где $0\leq x \leq M$, $0\leq y\leq N$ ($M$, $N$~--- ширина и высота изображения соответственно). Рассматриваем только полутоновые изображения со значениями яркости $0\leq f(x,y)\leq 1.0$. Пусть $f(x,y)$~--- неискажённое изображение; $g(x,y)$~--- изображение подвергнутое искажению; $h(x,y)$~--- импульсная характеристика (ИХ) оператора искажения; $\hat{f}(x,y)$~--- оценка неискажённого изображения $f(x,y)$; $\eta(x,y)$~--- некореллированный гауссов шум.
Обозначим $F(u,v), G(u,v), H(u,v), N(u,v)$ Фурье-образы функций $f(x,y), g(x,y), h(x,y)$ и $\eta(x,y)$ соответственно, полученные дискретным преобразованием Фурье \cite[стр.~284]{gonsalesDigital}




\subsection{Название подсекции}

Длинный-длинный абзац, в котором рассказывается про отношение пишущего все это человека к существующему мироустройству, порядку вещей, главному вопросу жизни, вселенной и всего такого.
